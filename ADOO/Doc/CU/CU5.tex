\begin{UseCase}{CU7}{Registrar Venta a Mostrador.}{
	El cajero debe tener una manera fácil de realizar una venta, la venta se realizará al publico en general, utilizando la pantalla de ventas, 
	donde el cajero tomará los datos de el o los medicamentos que se venderán, así como el número de unidades a adquirir teniendo en cuenta
	que no podrá vender más medicamento del que se tiene en el almacén, una vez realizada la venta el sistema generará un comprobante para el cliente
	el cual servirá para cualquier aclaración, duda o problema que pueda surgir y para fines de control y consulta por parte de la farmacia.
	}
		\UCitem{Versión}{\color{Gray}0.9}
		\UCitem{Autor}{\color{Gray}Correa Medina Carlos Miguel.}
		\UCitem{Supervisa}{\color{Gray}Vázquez Cruz Fernando Darwin.}
		\UCitem{Actor}{Cajero.}
		\UCitem{Propósito}{Tener un control de las salidas de medicamento por ventas a Mostrador, para fines de cualquier aclaración o duda y poder generar un reporte de ventas diarias.}
		\UCitem{Entradas}{Tipo de cliente,código de barras de lo(s) medicamento(s).}
		\UCitem{Origen}{Teclado, mouse, lector de código de barras.}
		\UCitem{Salidas}{Comprobante de venta con los datos nombre de sucursal, nombre completo del empleado,lista de medicamentos comprados, precio unitario de cada medicamento, valor total de la compra, cantidad a llevar de cada medicamento, fecha y hora de compra.}
		\UCitem{Destino}{Pantalla,impresora}
		\UCitem{Precondiciones}{Los medicamentos a vender deben estar registrados en el sistema, el supervisor debió abrir la caja con anterioridad \UCref{CU9}{Abrir Caja}, los medicamentos a vender deben tener fecha de caducidad vigente mayor a 2 meses.}
		\UCitem{Postcondiciones}{Disminuye la cantidad de los medicamentos que se vendieron, se tiene una venta a mostrador más registrada en el sistema , se aumenta el dinero existente en la caja.}
		\UCitem{Errores}{Que no se realice una conexión a la base de datos,que el servidor se caiga,que los datos proporcionados estén erróneos, que el o los medicamentos estén registrados en el sistema pero no estén disponibles en la sucursal, que no haya conexión con la impresora, que la impresora no tenga rollo.}
		\UCitem{viene de...}{\UCref{CU0}{Control de acceso}}
		\UCitem{Observaciones}{}
		\UCitem{Estado}{Aprobado}
	\end{UseCase}
%--------------------------------------
	\begin{UCtrayectoria}{Principal}
		\UCpaso Se incluye el caso de uso \UCref{CU0}.
		\UCpaso [\UCactor] Presiona el botón \IUbutton{Ventas} que esta en la pantalla principal \IUref{IU1}{Pantalla principal}
		\UCpaso Verifica que los permisos de usuario sean permisos de cajero. \Trayref{A}		
		\UCpaso Busca el id del cajero que esta actualmente con la sesión activa y llena el campo "Empleado" con el nombre completo del empleado.
		\UCpaso Genera y despliega el Formulario \IUref{IU19}{Formulario Ventas}.
		\UCpaso [\UCactor] En el primer campo de formulario "Cliente" Selecciona la opción "Cliente a Mostrador".
		\UCpaso Identifica la opción seleccionada por el [\UCactor].
		\UCpaso [\UCactor] Escanea el código de barras del medicamento con el lector de código de barras.
		\UCpaso Llena el campo de `"Medicamentos" con el código de barras que se introdujo del lector.\Trayref{C}.
		\UCpaso Llena el campo "Desglose de medicamentos, precios y cantidad" con el nombre del medicamento, el precio.
		\UCpaso Restringe la venta del medicamento a las unidades existentes a la sucursal.
		\UCpaso [\UCactor] En el campo "Desglose de medicamentos, precios y cantidad" selecciona la cantidad del medicamento a vender.\Trayref{D}
		\UCpaso Calcula el Total a pagar y muestra la cantidad en el campo "Valor Total".
		\UCpaso [\UCactor] Presiona el botón \IUbutton{Guardar}.
		\UCpaso Muestra el Mensaje {\bf MSG6-}"Confirmar[{\em Confirmar Operación }]¿Los datos del formulario son correctos?.".
		\UCpaso [\UCactor] Presiona el botón \IUbutton{Aceptar}
		\UCpaso Verifica que ningún campo del formulario esté vació. \Trayref{E}
		\UCpaso Guarda los datos de la venta y reduce el inventario del medicamento vendido.
		\UCpaso Muestra el Mensaje {\bf MSG0-}"Éxito[{\em Operación Realizada con éxito }].".
		\UCpaso Añade los datos de la venta al reporte diario de ventas.
		\UCpaso Muestra la pantalla \IUref{IU1}{Pantalla principal}.
	\end{UCtrayectoria}

%--------------------------------------		
	\begin{UCtrayectoriaA}{A}{Permiso Denegado}
			\UCpaso Muestra el Mensaje {\bf MSG4-}`"Cancelado[{\em Permiso Denegado }].".
			\UCpaso Muestra la pantalla \IUref{IU1}{Pantalla Principal}.
		\end{UCtrayectoriaA}


		%--------------------------------------
		\begin{UCtrayectoriaA}{C}{El lector de barras no pudo leer bien el código}
			\UCpaso Sigue su ejecución sin llenar el campo "Medicamentos""
			\UCpaso Continua en el paso 11 de este caso de uso.	
		\end{UCtrayectoriaA}
%----------------------------------------
		\begin{UCtrayectoriaA}{D}{El cliente quiere comprar más de 20 unidades del mismo medicamento}
			\UCpaso Muestra el Mensaje {\bf MSG4-}`"[{\em Permiso Denegado}] Contacta a tu supervisor.".
			\UCpaso [\UCactor] Solicita la autorización del supervisor.
			\UCpaso Regresa al paso 15 de este caso de uso.
		\end{UCtrayectoriaA}		
		%--------------------------------------
			\begin{UCtrayectoriaA}{E}{Existe por lo menos un campo obligatorio que esta vacío}
			\UCpaso Remarca con rojo los campos obligatorios que están vacíos y pone la leyenda "Campo Obligatorio".
			\UCpaso Muestra el Mensaje {\bf MSG1-}"Error en la operación [{\em Campos obligatorios vacíos}] Los campos llenados con Rojo no pueden estar vacíos.".
			\UCpaso[\UCactor] Oprime el botón \IUbutton{Aceptar}.
			\UCpaso Regresa a la pantalla \IUref{IU19}{Formulario Ventas} Mostrando los campos en rojo y dejando la información introducida por el [\UCactor] en los pasos anteriores de este casos de uso.
		\end{UCtrayectoriaA}
